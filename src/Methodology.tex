The specific objectives of all the subprojects are integrated in the PETALO Project Management Plan (PMP). The PMP coordinates the construction of P2 and P10 detectors. It is under the direct supervision of the Spokesperson (SP), the Medical Spokesperson (MSP), the Project Manager (PM) and the Medical Project Manger (MPM), and links with the PIs of this project\footnote{The PETALO SP is Prof. J.J. Gómez Cadenas; the MSP is Dr. Luis Martí-Bonmatí; the PM is Prof. J. Toledo; the MPM is Dr. Ángel Alberich. The collaboration steering committee includes the SP, MSP, PM, MPM and the PIs of the three projects presented here}. 

The PMP defines rigorously the activities (also called working packages WP) of the project and follows the progress of each one, monitors deliverables and deadlines and keeps track of invested resources including personnel. It also identifies potential show-stoppers and synergies (and possible conflicts) between the different projects and optimises the sharing of resources. 

%Figure \ref{fig.Gantt} shows an example of the Gantt chart for the whole NEW project, up to installaton at the LSC. 

%The objectives defined for the different sub projects match the various WP in the PMP, as can be seen in Figure \ref{Fig:PMP}. 

The methodology of each WP includes: a) the definition of the associated tasks; b) the identification of the resources needed; c) the temporal organisation of the tasks; d) the definition of milestones and the deliverables associated to them; e) the relations with other WP. Each WP has a leader, which reports directly to the PM/MPM. The progress of each WP is reviewed on a weekly basis. Milestones and potential showstoppers are discussed, and the tracking charts updated if needed. 