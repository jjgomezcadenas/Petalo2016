%C.3. IMPACTO ESPERADO DE LOS RESULTADOS 

The expected impact of this project is very large. A successful proof of concept of PETALO would open the way to the subsequent construction of pre-clinical (e.g, small animal PET) and clinical PETs. PETALO offers many advantages over conventional SSDs based devices, including a much better energy resolution, true 3D reconstruction that minimises parallax effects and the capability to handle Compton interactions, which are much better identified in the LXSC than in conventional SSDs. In addition, PETALO will be designed, from the beginning, as a fully compatible device with NMR, including both hardware components and the development of the imaging software. On top of the above, PETALO offers the potential of a breakthrough in TOF-PET technology. Last, but not least, the cost of the detection material (LXe) is much lower than the cost SSDs such as LSO and the cost of the SiPMs is decreasing exponentially. Therefore, by developing suitable front-end electronics and DAQ, PETALO could become a very economical PET solution, appropriated for a future large-scale (full-body) PET.

Each subproject in this coordinated project contributes in a decisive way to the impact of PETALO. Subproject DET focuses in transferring the technology developed by a basic science experiment (NEXT) to an application of enormous importance for public health. Subproject ASIC is essential, not only to provide the needed electronics for the proof of concept, but to develop the APE chip, which is mandatory for a large-scale application. The subproject IMG will, from the very beginning, focus in the integration of PETALO and NMR. 

PETALO has already produced a patent request, and a number of other patents will certainly follow. The apparatus can be clearly commercialised, given its many advantages over conventional PETs. The diffusion of the results will proceed through publications in journals and participation in national and international conferences. Once the proof of concept is operational, the research team will work actively with companies to explore join ventures and possibilities of transferring to the industrial sector. 
