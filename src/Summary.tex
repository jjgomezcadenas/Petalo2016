This project proposes a proof-of-concept for a new type of PET apparatus, called
PETALO (PET with TOF Applications based in Liquid XenOn). The active target is liquid xenon (LXe), read by silicon photomultipliers (SiPMs). The basic element of PETALO is the Liquid Xenon Scintillating Cell (LXSC). Its dimensions are $5\times 5 \times 5$~cm$^3$, optimised to maximise the number of gammas that interact in the cell and to minimise pileup. The best performance is obtained for a configuration where all the six faces are instrumented with matrices of  8$\times$8 SiPMs, each of $6 \times 6$~mm$^2$~area (LXSC6). The most economical configuration (LXSC2) instruments only the entry and exit faces (relative to the incoming gammas) but still displays an excellent performance. 

Xenon is a noble gas which scintillates as response to the ionising radiation. The scintillation is very fast (2.2 ns) and very intense (37,000 photons per 511 keV gamma). The combination of both features results in the possibility of building a PET of excellent resolution (better than 5\% FWHM, to be compared with 20\% typical of commercial devices) {\em and excellent time resolution}. This, in turn, makes it possible the application of PETALO as PET-TOF. The use of time of flight to reduce imaging errors is known since the beginning of the technology and the last few years have witnessed a rekindled interest in the technique, including the introduction in the market of a commercial model of 600 ps CRT (Coincidence Resolution Time). Other recent devices feature improved CRT near 400 ps. PETALO could achieve CRTs in the range of 200-250 ps, thus representing a break-through in the technology.  

In addition of the above advantages, PETALO will be designed from the very beginning as a detector compatible with Nuclear Magnetic Resonance (NMR), thus capable to operate in combination with this technology. 

Last but not least, given the low cost of xenon compared with that of conventional scintillators such as LSO, and the application of the modern SiPM technology (the costs of SiPMs has decreased dramatically over the last few years while their performance has continuously improved), it appears possible that PETALO could result in a low-cost system, suited for a full body PET.

This projects coordinates three groups whose combined experience permits the construction and commissioning of a demonstrator of PETALO based in 4 LXSC2, called P4. The DET subproject (IFIC) is in charge of the construction  of the detector, transferring to this very applied problem the technology developed in the context of a basic science enterprise, a neutrinoless double beta decay experiment called NEXT. It is also in charge of the development of the simulation and software framework, and will lead the study of PETALO as a TOF device. The ASIC subproject (I3M/UPV) is in charge of the development of the electronics to operate P4 as well as of the development of an ASIC optimised for the technology, called APE (Asic for PEtalo). The subproject IMG 
(GIBI239/LaFe) is in charge of image analysis, operation and certification of P4 as a NMR compatible device. 
 



