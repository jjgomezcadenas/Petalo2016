Este proyecto propone un programa de I+D+i cuyo objetivo es demostrar las prestaciones y viabilidad de un nuevo tipo de aparato PET, denominado PETALO (PET with TOF Applications based in Liquid XenOn). El material activo es xenón líquido leído por fotomultiplicadores de silicio (SiPM). El elemento básico de PETALO es la celda centelleadora de xenón líquido (LXSC), de dimensiones $5\times 5 \times 5$~cm$^3$, optimizada para maximizar el número de gammas que interaccione en la celda y minimizar el pile-up. La configuración con mayor rendimiento de la LXSC instrumenta sus seis caras internas con matrices de 8$\times$8 SiPM de $6 \times 6$~mm$^2$~de superficie (denominamos esta configuración LXSC6). La configuración más económica, que sin embargo mantiene una excelente resolución tanto en energía como en posición, instrumenta sólo la cara de entrada y de salida (relativa a la dirección de los fotones incidentes) y se denomina LXSC2.

El xenón es un gas noble, que centellea en respuesta a la radiación. La señal de centelleo del xenón líquido es muy rápida (2.2 ns) y muy intensa (37,000 fotones por gama de 511 keV). La combinación de ambas características hace posible utilizar el xenón líquido para construir un PET de gran resolución en energía (alrededor del 5\% FWHM, a comparar con el 20\% típico de los PET convencionales modernos) {\em y gran resolución temporal}, lo que permite su aplicación como PET-TOF. La aplicación de tiempo de vuelo para reducir los errores de reconstrucción de imagen en PET, conocida desde el albor de la tecnología ha resurgido con fuerza en los últimos años, en los que se han construido modelos comerciales con resoluciones temporales entre 400 y 600 ps que mejoran considerablemente la resolución de los PET convencionales. PETALO aspira a una resolución temporal en el rango de los 200-250 ps, lo que supondría un salto cuantitativo en la tecnología.

Además de las ventajas anteriores, PETALO se diseña desde el principio como un detector compatible con Resonancia Magnética Nuclear (RMN) y por tanto capaz de operar en combinación con este tecnología.

Por último, dado el bajo coste del xenón comparado con los centelleadores utilizados en los sistemas PET modernos (tales como el LSO) y el uso de la moderna tecnología de SiPMs, cuyo coste ha disminuido exponencialmente en los últimos años, la tecnología propuesta por PETALO podría resultar en un sistema lo bastante económico como para construir aparatos de gran tamaño (PET de cuerpo completo). 

Este proyecto coordina tres grupos cuya experiencia combinada permiten la construcción y puesta a punto de un demostrador de PETALO basado en cuatro LXSC2, llamado P4. El subproyecto DET (IFIC) está a cargo de la construcción del detector, aplicando la tecnología desarrollada en el IFIC en el contexto del experimento NEXT. También desarrollará la simulación y el entorno de software para la operación del detector y se ocupará del estudio de su capacidad como PET-TOF. El subproyecto ASIC (I3M/UPV) está a cargo del desarrollo de la electrónica para operar P4 así como del desarrollo de un ASIC, denominado APE (Asic for PEtalo) específico para esta tecnología. El subproyecto IMG (GIBI239/LaFe) está a cargo del análisis de imagen, la operación y certificación de P4 como instrumento compatible con RMN. 



